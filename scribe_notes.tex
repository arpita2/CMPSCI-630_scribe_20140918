\documentclass[twoside]{article}
\setlength{\oddsidemargin}{0.25 in}
\setlength{\evensidemargin}{-0.25 in}
\setlength{\topmargin}{-0.6 in}
\setlength{\textwidth}{6.5 in}
\setlength{\textheight}{8.5 in}
\setlength{\headsep}{0.75 in}
\setlength{\parindent}{0 in}
\setlength{\parskip}{0.1 in}

\usepackage{graphicx}
\usepackage{url}

%
% The following commands sets up the lecnum (lecture number)
% counter and make various numbering schemes work relative
% to the lecture number.
%
\newcounter{lecnum}
\renewcommand{\thepage}{\thelecnum-\arabic{page}}
\renewcommand{\thesection}{\thelecnum.\arabic{section}}
\renewcommand{\theequation}{\thelecnum.\arabic{equation}}
\renewcommand{\thefigure}{\thelecnum.\arabic{figure}}
\renewcommand{\thetable}{\thelecnum.\arabic{table}}
\newcommand{\dnl}{\mbox{}\par}

%
% The following macro is used to generate the header.
%
\newcommand{\lecture}[4]{
  \pagestyle{myheadings}
  \thispagestyle{plain}
  \newpage
  \setcounter{lecnum}{#1}
  \setcounter{page}{1}
  \noindent
  \begin{center}
  \framebox{
     \vbox{\vspace{2mm}
   \hbox to 6.28in { {\bf CMPSCI~377~~~Operating Systems
                       \hfill Fall 2009} }
      \vspace{4mm}
      \hbox to 6.28in { {\Large \hfill Lecture #1  \hfill} }
%       \hbox to 6.28in { {\Large \hfill Lecture #1: #2  \hfill} }
      \vspace{2mm}
      \hbox to 6.28in { {\it Lecturer: #3 \hfill Scribe: #4} }
     \vspace{2mm}}
  }
  \end{center}
  \markboth{Lecture #1: #2}{Lecture #1: #2}
  \vspace*{4mm}
}

%
% Convention for citations is authors' initials followed by the year.
% For example, to cite a paper by Leighton and Maggs you would type
% \cite{LM89}, and to cite a paper by Strassen you would type \cite{S69}.
% (To avoid bibliography problems, for now we redefine the \cite command.)
%
\renewcommand{\cite}[1]{[#1]}

% \input{epsf}

%Use this command for a figure; it puts a figure in wherever you want it.
%usage: \fig{NUMBER}{FIGURE-SIZE}{CAPTION}{FILENAME}
\newcommand{\fig}[4]{
           \vspace{0.2 in}
           \setlength{\epsfxsize}{#2}
           \centerline{\epsfbox{#4}}
           \begin{center}
           Figure \thelecnum.#1:~#3
           \end{center}
   }

% Use these for theorems, lemmas, proofs, etc.
\newtheorem{theorem}{Theorem}[lecnum]
\newtheorem{lemma}[theorem]{Lemma}
\newtheorem{proposition}[theorem]{Proposition}
\newtheorem{claim}[theorem]{Claim}
\newtheorem{corollary}[theorem]{Corollary}
\newtheorem{definition}[theorem]{Definition}
\newenvironment{proof}{{\bf Proof:}}{\hfill\rule{2mm}{2mm}}

% Some useful equation alignment commands, borrowed from TeX
\makeatletter
\def\eqalign#1{\,\vcenter{\openup\jot\m@th
 \ialign{\strut\hfil$\displaystyle{##}$&$\displaystyle{{}##}$\hfil
     \crcr#1\crcr}}\,}
\def\eqalignno#1{\displ@y \tabskip\@centering
 \halign to\displaywidth{\hfil$\displaystyle{##}$\tabskip\z@skip
   &$\displaystyle{{}##}$\hfil\tabskip\@centering
   &\llap{$##$}\tabskip\z@skip\crcr
   #1\crcr}}
\def\leqalignno#1{\displ@y \tabskip\@centering
 \halign to\displaywidth{\hfil$\displaystyle{##}$\tabskip\z@skip
   &$\displaystyle{{}##}$\hfil\tabskip\@centering
   &\kern-\displaywidth\rlap{$##$}\tabskip\displaywidth\crcr
   #1\crcr}}
\makeatother

% **** IF YOU WANT TO DEFINE ADDITIONAL MACROS FOR YOURSELF, PUT THEM HERE:



% Some general latex examples and examples making use of the
% macros follow.

\begin{document}

%FILL IN THE RIGHT INFO.
%\lecture{**LECTURE-NUMBER**}{**DATE**}{**LECTURER**}{**SCRIBE**}
\lecture{17}{April 16}{Emery Berger}{Bruno Silva,Jim Partan}

\section{Review of Custom Allocators}

We reviewed per-class allocators, custom pattern allocators, and
region allocators from last lecture.

\section{DieHard Allocator}

Allocators can also be used to avoid problems with unsafe languages.
C and C++ are pervasive, with huge amounts of existing code. They are
also memory-unsafe languages, in that they allow many errors and
security vulnerabilities. Some examples include double {\tt free()},
invalid {\tt free()}, uninitialized reads, dangling pointers,
and buffer overflows in both stack and heap buffers.

DieHard is an allocator developed at UMass which provides (or at least
improves) soundness for erroneous programs.

There are several hardware
trends which are occurring: multicore processors are becoming the norm,
physical memory is relatively inexpensive, and \mbox{64-bit} architectures
are increasingly common, with huge virtual address spaces. Meanwhile,
most programs have trouble making full use of multiple processors.
The net result is that there may soon be unused processing power and
enormous virtual address spaces.

If you had an infinite address space,
you wouldn't have to worry about freeing objects. That would mostly eliminate
the double {\tt free()}, invalid {\tt free()}, and dangling pointer bugs.
And if your heap objects were infinitely far apart in memory, you wouldn't
need to worry about buffer overflows in heap objects.

DieHard tries to provide something along these lines, within the constraints
of finite physical memory. It uses randomized heap allocation, so objects
are not necessarily contiguous in virtual memory. Since the address space
is actually finite, objects won't actually be infinitely far apart, and
buffer overruns might actually cause collisions between heap objects. But this
is where the multicore processors come in: With the unused processor cores,
run multiple copies of the application, say three copies,
each allocating into their own randomized heap. So the heap errors are
independent among the three copies of the application.
All copies get the same input,
and the output is the result of voting among the three copies of the program.
If one instance of the application disagrees with the other two, it is killed,
since there was likely a collision between heap objects in that one. Similarly,
if one instance dies with a segfault or other error, the others remain running.
Surviving copies can be forked to replace copies which were killed off, though
this reduces the independence among copies.

This is transparent to correct applications, and allows erroneous applications
to survive longer, which will make their users happier. Furthermore, when
a copy dies or is killed off, the error can be noted and sent to the software
maintainers automatically.

DieHard increases the chances of turning problematic errors into
benign errors. It trades memory space for robustness, which is probably
fine as memory prices continue to drop. It uses randomization and
replication to provide fault-tolerance.

\end{document}
